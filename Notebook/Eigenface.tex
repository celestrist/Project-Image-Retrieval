%% This file was auto-generated by IPython, do NOT edit
%% Conversion from the original notebook file:
%% Eigenface.ipynb
%%
\documentclass[11pt,english]{article}

%% This is the automatic preamble used by IPython.  Note that it does *not*
%% include a documentclass declaration, that is added at runtime to the overall
%% document.

\usepackage{amsmath}
\usepackage{amssymb}
\usepackage{graphicx}
\usepackage{ucs}
\usepackage[utf8x]{inputenc}

% needed for markdown enumerations to work
\usepackage{enumerate}

% Slightly bigger margins than the latex defaults
\usepackage{geometry}
\geometry{verbose,tmargin=3cm,bmargin=3cm,lmargin=2.5cm,rmargin=2.5cm}

% Define a few colors for use in code, links and cell shading
\usepackage{color}
\definecolor{orange}{cmyk}{0,0.4,0.8,0.2}
\definecolor{darkorange}{rgb}{.71,0.21,0.01}
\definecolor{darkgreen}{rgb}{.12,.54,.11}
\definecolor{myteal}{rgb}{.26, .44, .56}
\definecolor{gray}{gray}{0.45}
\definecolor{lightgray}{gray}{.95}
\definecolor{mediumgray}{gray}{.8}
\definecolor{inputbackground}{rgb}{.95, .95, .85}
\definecolor{outputbackground}{rgb}{.95, .95, .95}
\definecolor{traceback}{rgb}{1, .95, .95}

% Framed environments for code cells (inputs, outputs, errors, ...).  The
% various uses of \unskip (or not) at the end were fine-tuned by hand, so don't
% randomly change them unless you're sure of the effect it will have.
\usepackage{framed}

% remove extraneous vertical space in boxes
\setlength\fboxsep{0pt}

% codecell is the whole input+output set of blocks that a Code cell can
% generate.

% TODO: unfortunately, it seems that using a framed codecell environment breaks
% the ability of the frames inside of it to be broken across pages.  This
% causes at least the problem of having lots of empty space at the bottom of
% pages as new frames are moved to the next page, and if a single frame is too
% long to fit on a page, will completely stop latex from compiling the
% document.  So unless we figure out a solution to this, we'll have to instead
% leave the codecell env. as empty.  I'm keeping the original codecell
% definition here (a thin vertical bar) for reference, in case we find a
% solution to the page break issue.

%% \newenvironment{codecell}{%
%%     \def\FrameCommand{\color{mediumgray} \vrule width 1pt \hspace{5pt}}%
%%    \MakeFramed{\vspace{-0.5em}}}
%%  {\unskip\endMakeFramed}

% For now, make this a no-op...
\newenvironment{codecell}{}

 \newenvironment{codeinput}{%
   \def\FrameCommand{\colorbox{inputbackground}}%
   \MakeFramed{\advance\hsize-\width \FrameRestore}}
 {\unskip\endMakeFramed}

\newenvironment{codeoutput}{%
   \def\FrameCommand{\colorbox{outputbackground}}%
   \vspace{-1.4em}
   \MakeFramed{\advance\hsize-\width \FrameRestore}}
 {\unskip\medskip\endMakeFramed}

\newenvironment{traceback}{%
   \def\FrameCommand{\colorbox{traceback}}%
   \MakeFramed{\advance\hsize-\width \FrameRestore}}
 {\endMakeFramed}

% Use and configure listings package for nicely formatted code
\usepackage{listingsutf8}
\lstset{
  language=python,
  inputencoding=utf8x,
  extendedchars=\true,
  aboveskip=\smallskipamount,
  belowskip=\smallskipamount,
  xleftmargin=2mm,
  breaklines=true,
  basicstyle=\small \ttfamily,
  showstringspaces=false,
  keywordstyle=\color{blue}\bfseries,
  commentstyle=\color{myteal},
  stringstyle=\color{darkgreen},
  identifierstyle=\color{darkorange},
  columns=fullflexible,  % tighter character kerning, like verb
}

% The hyperref package gives us a pdf with properly built
% internal navigation ('pdf bookmarks' for the table of contents,
% internal cross-reference links, web links for URLs, etc.)
\usepackage{hyperref}
\hypersetup{
  breaklinks=true,  % so long urls are correctly broken across lines
  colorlinks=true,
  urlcolor=blue,
  linkcolor=darkorange,
  citecolor=darkgreen,
  }

% hardcode size of all verbatim environments to be a bit smaller
\makeatletter 
\g@addto@macro\@verbatim\small\topsep=0.5em\partopsep=0pt
\makeatother 

% Prevent overflowing lines due to urls and other hard-to-break entities.
\sloppy

\begin{document}

Appendix B: Python Demonstration on Using PCA and SVM in Face Recognition
\begin{codecell}
\begin{codeinput}
\begin{lstlisting}
print __doc__

from time import time
import logging
import pylab as pl

from sklearn.cross_validation import train_test_split
from sklearn.datasets import fetch_lfw_people
from sklearn.grid_search import GridSearchCV
from sklearn.metrics import classification_report
from sklearn.metrics import confusion_matrix
from sklearn.decomposition import RandomizedPCA
from sklearn.svm import SVC

# Display progress logs on stdout
logging.basicConfig(level=logging.INFO, format='%(asctime)s %(message)s')


###############################################################################
# Download the data, if not already on disk and load it as numpy arrays

lfw_people = fetch_lfw_people(min_faces_per_person=70, resize=0.4)

# introspect the images arrays to find the shapes (for plotting)
n_samples, h, w = lfw_people.images.shape

# fot machine learning we use the 2 data directly (as relative pixel
# positions info is ignored by this model)
X = lfw_people.data
n_features = X.shape[1]

# the label to predict is the id of the person
y = lfw_people.target
target_names = lfw_people.target_names
n_classes = target_names.shape[0]

print "Total dataset size:"
print "n_samples: %d" % n_samples
print "n_features: %d" % n_features
print "n_classes: %d" % n_classes


###############################################################################
# Split into a training set and a test set using a stratified k fold

# split into a training and testing set
X_train, X_test, y_train, y_test = train_test_split(
    X, y, test_size=0.25)


###############################################################################
# Compute a PCA (eigenfaces) on the face dataset (treated as unlabeled
# dataset): unsupervised feature extraction / dimensionality reduction
n_components = 150

print "Extracting the top %d eigenfaces from %d faces" % (
    n_components, X_train.shape[0])
t0 = time()
pca = RandomizedPCA(n_components=n_components, whiten=True).fit(X_train)
print "done in %0.3fs" % (time() - t0)

eigenfaces = pca.components_.reshape((n_components, h, w))

print "Projecting the input data on the eigenfaces orthonormal basis"
t0 = time()
X_train_pca = pca.transform(X_train)
X_test_pca = pca.transform(X_test)
print "done in %0.3fs" % (time() - t0)


###############################################################################
# Train a SVM classification model

print "Fitting the classifier to the training set"
t0 = time()
param_grid = {
 'C': [1e3, 5e3, 1e4, 5e4, 1e5],
 'gamma': [0.0001, 0.0005, 0.001, 0.005, 0.01, 0.1],
}
clf = GridSearchCV(SVC(kernel='rbf', class_weight='auto'), param_grid)
clf = clf.fit(X_train_pca, y_train)
print "done in %0.3fs" % (time() - t0)
print "Best estimator found by grid search:"
print clf.best_estimator_


###############################################################################
# Quantitative evaluation of the model quality on the test set

print "Predicting the people names on the testing set"
t0 = time()
y_pred = clf.predict(X_test_pca)
print "done in %0.3fs" % (time() - t0)

print classification_report(y_test, y_pred, target_names=target_names)
print confusion_matrix(y_test, y_pred, labels=range(n_classes))


###############################################################################
# Qualitative evaluation of the predictions using matplotlib

def plot_gallery(images, titles, h, w, n_row=3, n_col=4):
    """Helper function to plot a gallery of portraits"""
    pl.figure(figsize=(1.8 * n_col, 2.4 * n_row))
    pl.subplots_adjust(bottom=0, left=.01, right=.99, top=.90, hspace=.35)
    for i in range(n_row * n_col):
        pl.subplot(n_row, n_col, i + 1)
        pl.imshow(images[i].reshape((h, w)), cmap=pl.cm.gray)
        pl.title(titles[i], size=12)
        pl.xticks(())
        pl.yticks(())


# plot the result of the prediction on a portion of the test set

def title(y_pred, y_test, target_names, i):
    pred_name = target_names[y_pred[i]].rsplit(' ', 1)[-1]
    true_name = target_names[y_test[i]].rsplit(' ', 1)[-1]
    return 'predicted: %s\ntrue:      %s' % (pred_name, true_name)

prediction_titles = [title(y_pred, y_test, target_names, i)
                     for i in range(y_pred.shape[0])]

plot_gallery(X_test, prediction_titles, h, w)

# plot the gallery of the most significative eigenfaces

eigenface_titles = ["eigenface %d" % i for i in range(eigenfaces.shape[0])]
plot_gallery(eigenfaces, eigenface_titles, h, w)

pl.show()
\end{lstlisting}
\end{codeinput}
\begin{codeoutput}
\begin{verbatim}

\end{verbatim}
\begin{verbatim}
Built-in functions, exceptions, and other objects.

Noteworthy: None is the `nil' object; Ellipsis represents `...' in slices.
Total dataset size:
\end{verbatim}
\begin{verbatim}
n_samples: 1288
n_features: 1850
n_classes: 7
Extracting the top 150 eigenfaces from 966 faces
done in 0.489s
\end{verbatim}
\begin{verbatim}
Projecting the input data on the eigenfaces orthonormal basis
done in 0.054s
\end{verbatim}
\begin{verbatim}
Fitting the classifier to the training set
done in 17.447s
\end{verbatim}
\begin{verbatim}
Best estimator found by grid search:
SVC(C=1000.0, cache_size=200, class_weight=auto, coef0=0.0, degree=3,
  gamma=0.005, kernel=rbf, probability=False, shrinking=True, tol=0.001,
  verbose=False)
Predicting the people names on the testing set
done in 0.054s
\end{verbatim}
\begin{verbatim}
precision    recall  f1-score   support

     Ariel Sharon       1.00      0.61      0.76        18
     Colin Powell       0.82      0.92      0.87        64
  Donald Rumsfeld       0.95      0.70      0.81        27
    George W Bush       0.80      0.95      0.87       124
Gerhard Schroeder       0.96      0.75      0.84        36
      Hugo Chavez       1.00      0.65      0.79        23
       Tony Blair       0.90      0.87      0.88        30

      avg / total       0.87      0.85      0.85       322

[[ 11   4   1   2   0   0   0]
 [  0  59   0   5   0   0   0]
 [  0   1  19   7   0   0   0]
 [  0   4   0 118   1   0   1]
 [  0   0   0   7  27   0   2]
 [  0   4   0   4   0  15   0]
 [  0   0   0   4   0   0  26]]
\end{verbatim}
\begin{center}
\includegraphics[width=6in]{Eigenface_files/Eigenface_fig_00.png}
\par
\end{center}
\begin{center}
\includegraphics[width=6in]{Eigenface_files/Eigenface_fig_01.png}
\par
\end{center}
\end{codeoutput}
\end{codecell}

\end{document}
